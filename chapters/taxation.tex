\subsection{Does SARS regard cryptocurrency as a real currency?}

No - SARS does not regard cryptocurrency as a currency, but rather as an intangible asset.  In fact, the annual tax return (ITR12) refers to 'Crypto Assets' and not Cryptocurrency.  As there are currently no laws or regulations governing the use of cryptocurrency in South Africa, crypto traders or users have limited legal protection according to the general common law. In other words, if you choose to use or trade cryptocurrency you do so at your own risk.


\subsection{How does SARS treat crypto?}


SARS has made it clear that crypto transactions  will be taxed according to the existing South African tax laws. This means that crypto profits will either be taxed based on capital gains tax principals, or as revenue transactions (i.e as normal income, like your salary or freelance income). This will depend on the situation of the taxpayer and their intention for purchasing the crypto in the first place. 


\subsection{Are crypto traders subject to tax based on the income they earn?}


Currently, profits earned on crypto trading is subject to normal tax. If the total taxable profits earned is higher than the tax threshold for that particular financial year, then the taxpayer will be liable to register as a provisional tax payer.


\subsection{If I receive cryptocurrency as payment for goods or services, will I be taxed on it?}


Yes, this type of income will be subject to normal tax.

 

\subsection{If I am paid in crypto for contract work performed, will I be taxed on it when I receive it? Or only when I sell it one day?}


Yes, the crypto you receive will be subject to normal tax. If the total taxable income earned is higher than the tax threshold for that particular financial year, then you will need to register as a provisional taxpayer. You will be able to deduct business related expenses against the crypto income, in exactly the same way that you can deduct business expenses against other independent contractor / freelance work. When you later sell the crypto asset for fiat (i.e non-digital, 'normal' currency), you will have to pay  tax on any increase in value from the date you received the crypto to the date of sale. The tax will be calculated on revenue or capital gains tax principals, dependent on your circumstances.


\subsection{If my employer pays my salary in crypto, how will I be taxed?}

This type of income will be treated just like other remuneration and will be subject to normal tax at the earlier of receipt or accrual. It is important to point out that when you convert the crypto to fiat currency at a later date, this will trigger another taxable event and you will need to declare the gain/loss in your tax return.  This will be the change in value from the date you originally received it as salary, to what you eventually sell it for. 

\subsection{Will I be subject to penalties if I don't treat crypto transactions in a way that complies with South African tax laws?}

Yes, you may be subject to penalties for all act of tax non-compliance, crypto included!


\subsection{What would the acceptable proof of purchase and sale price be for cryptocurrencies?}

Conventional receipts, invoices and/or trading statements from the relevant exchange would be considered acceptable.


\subsection{When crypto is sold, what method is used to work out the cost? LIFO, FIFO or weighted average cost?}

SARS has made it clear that cryptocurrency is not considered a share, which means the average cost cannot be used. The correct approach is to use the first in first out (FIFO) method. 

Let's look at a quick example: 

2019: John buys 1 BTC for R100 

2020: John buys 1 BTC for R200

2021: John sells 1 BTC for R400

Based on FIFO, the cost of the BTC sold will be R100 (i.e  1 BTC purchased in 2019).

The gain will therefore be R300 (R400-R100). This gain will either be treated as revenue and added to the taxpayer's taxable income, or it will be taxed as a capital gain. This will depend on the circumstances of the transaction and the taxpayer's intention when they bought the BTC. In their latest guide however,  SARS did mention that due to their extreme volatility, crypto assets are most likely to be purchased for speculative purposes and therefore crypto transactions would be likely to be of a revenue (i.e trading) nature.

If I trade one crypto asset for another crypto asset, do I need to pay tax?

Yes, there will be a taxable event at the date of the asset swap.

Let's look at an example:

2019: Nina buys 1 BTC for R100 

2020: Nina buys 1 BTC for R200

2021: Nina sells 1 BTC for 2 ETH. The market value on this date is 1 BTC = R350

The cost of the BTC sold will be R100. There will therefore be a taxable gain of R250 (R350-R100) which will be taxed according to revenue or capital gains principals. It's important to note that it's the sale of the BTC which results in the taxable event, and not the purchase of the ETH. The tax on the ETH transaction will happen only later when it is converted to fiat currency.

\subsection{What if I receive crypto assets for mining or staking - how does the tax work?}

The receipt of these assets must be treated as revenue and will be taxed accordingly. This means if you mine and receive 1 BTC on 31 January 2022 which has a market value of R200 at this date, you must add R200 to your taxable income in your 2022 tax return.  If you later sell the 1 BTC for R400, this will trigger another taxable event and you will need to report the R200 gain (R400-R200) in your tax return as either a revenue or capital gain transaction depending on your intention and circumstances of the disposal.

\subsection{How are air drops and rewards treated for tax?}

Air drops are a marketing strategy to raise awareness about a new crypto asset. It involves sending crypto assets for free to numerous wallet addresses. There is no clear guidance as to how this type of transaction should be treated for tax, but it makes sense to treat  the same way as crypto assets received for mining (see previous FAQ).

\subsection{I am earning a yield (i.e interest) on the storage of my crypto assets.  Is this taxable?}

Yes, this is just like earning interest on a bank account and therefore would be taxable. It should be treated the same way as Bitcoin earned from mining. Because exchanges don't issue IT3bs for this 'interest', the interest exemption  (i.e R23 800 for under 65 year olds) would not apply.

\subsection{How will SARS track cryptocurrencies like Bitcoin?}

SARS has many powers as stipulated in the Income Tax Act. These enforcement and auditing processes are highly confidential and therefore not shared with the public. However, in June 2021, the three largest South African exchanges (AltCoinTrader, Luno and VALR) confirmed that they were approached by SARS to provide information on a selection of customers, which they were obliged by law to do.


\subsection{Will expenses incurred in generating Bitcoin income be tax-deductible?}

If you treat your crypto activity as revenue transactions (i.e trading) and declare in the local business section of your tax return,  then yes, all business expenses incurred in generating the crypto income would be tax-deductible.

\subsection{How should I declare cryptocurrency trading on my IRP6 (i.e Provisional Tax Return)?}

Profits from cryptocurrency trading (such as Bitcoin) should be declared in the total taxable income for the financial year.

\subsection{How and where on the ITR12 should I declare my cryptocurrency income?}

Your cryptocurrency profit will either be taxed as a capital gain or it will be subject to normal tax, depending on the specific details of the case. If it is taxed as a capital gain, you would need to declare in the capital gains section of the ITR12. If it is taxed as normal income, you would declare the income and related expenses in the local business section of the ITR12. 
